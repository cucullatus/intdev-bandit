\documentclass{sigchi}

% Use this command to override the default ACM copyright statement (e.g. for preprints). 
% Consult the conference website for the camera-ready copyright statement.


%% EXAMPLE BEGIN -- HOW TO OVERRIDE THE DEFAULT COPYRIGHT STRIP -- (July 22, 2013 - Paul Baumann)
% \toappear{Permission to make digital or hard copies of all or part of this work for personal or classroom use is 	granted without fee provided that copies are not made or distributed for profit or commercial advantage and that copies bear this notice and the full citation on the first page. Copyrights for components of this work owned by others than ACM must be honored. Abstracting with credit is permitted. To copy otherwise, or republish, to post on servers or to redistribute to lists, requires prior specific permission and/or a fee. Request permissions from permissions@acm.org. \\
% {\emph{CHI'14}}, April 26--May 1, 2014, Toronto, Canada. \\
% Copyright \copyright~2014 ACM ISBN/14/04...\$15.00. \\
% DOI string from ACM form confirmation}
%% EXAMPLE END -- HOW TO OVERRIDE THE DEFAULT COPYRIGHT STRIP -- (July 22, 2013 - Paul Baumann)


% Arabic page numbers for submission. 
% Remove this line to eliminate page numbers for the camera ready copy
% \pagenumbering{arabic}


% Load basic packages
\usepackage{balance}  % to better equalize the last page
\usepackage{graphics} % for EPS, load graphicx instead
\usepackage{times}    % comment if you want LaTeX's default font
\usepackage{url}      % llt: nicely formatted URLs

% llt: Define a global style for URLs, rather that the default one
\makeatletter
\def\url@leostyle{%
  \@ifundefined{selectfont}{\def\UrlFont{\sf}}{\def\UrlFont{\small\bf\ttfamily}}}
\makeatother
\urlstyle{leo}


% To make various LaTeX processors do the right thing with page size.
\def\pprw{8.5in}
\def\pprh{11in}
\special{papersize=\pprw,\pprh}
\setlength{\paperwidth}{\pprw}
\setlength{\paperheight}{\pprh}
\setlength{\pdfpagewidth}{\pprw}
\setlength{\pdfpageheight}{\pprh}

% Make sure hyperref comes last of your loaded packages, 
% to give it a fighting chance of not being over-written, 
% since its job is to redefine many LaTeX commands.
\usepackage[pdftex]{hyperref}
\hypersetup{
pdftitle={SIGCHI Conference Proceedings Format},
pdfauthor={LaTeX},
pdfkeywords={SIGCHI, proceedings, archival format},
bookmarksnumbered,
pdfstartview={FitH},
colorlinks,
citecolor=black,
filecolor=black,
linkcolor=black,
urlcolor=black,
breaklinks=true,
}

% create a shortcut to typeset table headings
\newcommand\tabhead[1]{\small\textbf{#1}}


% End of preamble. Here it comes the document.
\begin{document}

\title{FootGest: Using Shoes as Input Devices}

\numberofauthors{5}
\author{
  \alignauthor William Colaluca\\
    \affaddr{University of Bristol}\\
    \affaddr{Bristol, UK}\\
    \email{wc0941@bristol.ac.uk}\\
  \alignauthor Stephen de Mora\\
    \affaddr{University of Bristol}\\
    \affaddr{Bristol, UK}\\
    \email{sd1273@bristol.ac.uk}\\
  \alignauthor Nicholas Phillips\\
    \affaddr{University of Bristol}\\
    \affaddr{Bristol, UK}\\
    \email{np1970@bristol.ac.uk}\\
  \alignauthor James Savage\\
    \affaddr{University of Bristol}\\
    \affaddr{Bristol, UK}\\
    \email{js1999@bristol.ac.uk}\\
  \alignauthor Ashley Whetter\\
    \affaddr{University of Bristol}\\
    \affaddr{Bristol, UK}\\
    \email{aw0455@bristol.ac.uk}\\
}

\maketitle

\begin{abstract}
Feet are often overlooked as a valid medium for input. To address this, we have developed FootGest, a pair of sensor-embedded shoes and a gesture recognition application. In this paper we describe how this system was developed, along with analysing its use as an input device.
\end{abstract}

\keywords{
	Gesture recognition; Machine Learning; Shoes; Gyroscope; Accelerometer
}

\section{Introduction}
Address issue

State our solution

\section{Related Work}
blah blah Joe Paradiso blah blah Bristol PhD guy (maybe)

\section{Prototype Design}
blah
\subsection{Hardware}
blah
\subsection{Software}
blah
\subsubsection{Discrete Recognition (Machine Learning)}
blah
\subsubsection{Continuous Recognition}
blah

\section{Applications}
Disability, motion capture, general use (change music)

\section{Discussion and Future Work}
blah blah

\section{Conclusion}
blah blah

% REFERENCES FORMAT
% References must be the same font size as other body text.
\nocite{*}
\bibliographystyle{acm-sigchi}
\bibliography{refs}
\end{document}
